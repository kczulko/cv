\documentclass[a4paper,11pt]{cv4tw}% options are inherited from article class, and automatically given to this class
%% \usetheme[optional scheme]{theme}
\usetheme[minuit]{sharp}
%read env variables with secrets
\usepackage{catchfile}
\newcommand{\getenv}[2][]{%
  \CatchFileEdef{\temp}{"|kpsewhich --var-value #2"}{}%
  \ifx\temp\empty\def\temp{UNDEFINED}\fi
  \if\relax\detokenize{#1}\relax\temp\else\let#1\temp\fi}
% Not happy with the default style, or bullet ? Define yours !
\setmystyle[missionbullet]{\color{soft}$>$~~}
\setmystyle[assetbullet]{\color{icons}\faBook}
\setmystyle[fieldratefull]{\color{icons}\faCircle}
\setmystyle[fieldrateempty]{\color{icons}\faCircleO}
% If you need a custom contact, you can define a bullet here :
\setmystyle[custom0bullet]{\color{icons}\faBolt}
\usepackage[utf8]{inputenc}
%\academictitle{PhD}
\firstname{Karol}
\lastname{Czułkowski}
\title{scala developer}
\quote{\itshape Self employed scala developer with 10 years of overall experience}

%\cvpicture{./samplepic.jpg}
% Add up to 10 assets (0 to 9)
\setcvasset[0]{\emph{Experienced} software developer}
\setcvasset[1]{\emph{Functional} Programming}
\setcvasset[2]{\emph{Knowledge} sharing}
\setcvasset[3]{\emph{Team} player}
\setcvcontact[email]{karol.czulkowski@gmail.com}
\setcvcontact[cellphone]{\getenv{CELLPHONE}}
\setcvcontact[linkedin]{karol-czulkowski}
\setcvcontact[twitter]{@kczulko}
\setcvcontact[github]{kczulko}
\setcvcontact[spokenlanguages]{pl, en}
\setcvcontact[age]{\getenv{DATE_OF_BIRTH}}
\setcvcontact[address]{
  \getenv{ADDRESS_STREET}\\\getenv{ADDRESS_ZIP_CITY}\\\getenv{ADDRESS_COUNTRY}
}
% colors for \hyperref urls
\hypersetup{
  colorlinks = false,
  breaklinks = true,
  linkbordercolor = green,
  urlbordercolor = {0 1 1}
}
\begin{document}
\section{Experience}
%----------------------------------------------------------------------------
\story{scala developer}
      {\href{https://www.jit.team}{Jit Team} (under contract to Nordea Poland)}
      {since 07.2019\\Gdynia, Poland}
      {scala, cats, http4s,\\shapeless, tagless-final,\\fs2, bazel, nix, kafka,\\elasticsearch,\\microservices
      }
      {Development of HDFS metadata indexing hub (proprietary solution akin to \href{https://egeria.odpi.org}{egeria}).
        Product I work on has service oriented architecture build on the top of
        Elasticsearch. It allows datascientists to search for any data that exists
        within Nordea's datalake.
        Please find some of my achievements below:
        \begin{missions}
        \item I've introduced and teached internal workers modern FP architecture
        \item I've unified technology stack into more FP oriented one (for better testability)
        \item Implemented and refactored most of microservices
        \item Planned, designed and delivered several features
        \end{missions}
      }
%----------------------------------------------------------------------------
\story{scala developer}
      {\href{https://www.intel.com/content/www/us/en/jobs/locations/poland.html}{Intel Technology Poland}}
      {07.2018 - 06.2019\\Gdańsk, Poland}
      {compilers, shapeless,\\scala-macros, fs2, zio,\\monocle, scalaz,\\antlr4, sbt}
      {Work for pre-silicon group under Intel Network Division (before aquisition it was Fulcrum Microsystems from Caltech).
        Essentially, company hired me to develop scala simulation model for next generation ethernet switch.
        Due to the requirements change within organisation, this activity was abandoned.
        Since Network Divition group lacked good, modern compiler support for presilicon memory design activities, we decided to focus on
        \href{https://www.accellera.org/downloads/standards/systemrdl}{SystemRDL 1.0} (SRDL) compiler development.
        Here are some of my key achievements:
        \begin{missions}
        \item I've developed probably first fully compliant, built-in SRDL1.0 preprocessor
        \item Redefined SRDL1.0 grammar (official one had bugs)
        \item Designed and implemented purely functional memory model for next-gen Intel's ethernet switch simulator
        \end{missions}
      }
%----------------------------------------------------------------------------
\story{senior java developer}
      {\href{https://diversecg.pl/}{DCG} (under contract to Nordea Capital Markets)}
      {March - June, 2018\\Gdańsk, Poland}
      {java, spark, docker\\java-testcontainers}
      {I have joined the team responsible for Market Liquidity Credit Risk (MLQR) app development.
        MLQR was internal, proprietary application, which main activity was calulating daily - executed as batch job - credit risk coefficient values.
        Please find below some of my duties \& achievements:
        \begin{missions}
        \item Developed local spark deployment of ETL job for test configuration
        \item Code reviews \& bug fixing
        \end{missions}
      }
%----------------------------------------------------------------------------
\story{scala developer}{\href{https://www.scalac.io}{Scalac}}
      {2017 - 2018\\Gdańsk, Poland}
      {scala, akka-streams,\\graphql, cassandra,\\kafka, play, mtl,\\marathon}
      {For the vast majority of time spend at Scalac, my contribution was related with Qvantel - a Finnish business solution systems provider for telco companies.
        As a Scalac, we developed for Qvantel several microservices, mostly responsible for ETL jobs (messages transformation and ingestion).
        Scallable apps developed Qvantel were deployed on emerging markets like Bolivia, Equador or Honduras. For one of them we satisfied client's constraint
        to process \approx12k messages per second. More about this cooperation can be found
        \href{https://issuu.com/outlookpublishing/docs/qvantel}{\emph{>>here<<}}. Next, the team was asked to develop proof-of-concept application
        utilizing graphQL approach in order to get rid of plenty RESTful services and improve Qvantel's business.
      }
%----------------------------------------------------------------------------
\story{software engineer}
      {\href{https://www.intel.com/content/www/us/en/jobs/locations/poland.html}
        {Intel Technology Poland}}
      {2014 - 2017\\Gdańsk, Poland}
      {java ee, c\#, c++11,\\gradle, docker,\\nodejs, bash}
      {The company was looking for experienced validation engineer for newly formed \href{https://www.intel.com/intelRSD}{Rackscale Design (RSD)} team.
        My duties were to develop high quality testing libraries and provide mentorship for other team members.
        After some time I've joined development teams. Firstly, to help with firmware code, especially to model LVM system under firmware component.
        Later, I joined development team of
        \href{https://www.intel.com/content/www/us/en/architecture-and-technology/rack-scale-design/pod-manager-user-guide-v2-1.html}{POD Manager} component,
        where I spend most of my time at Intel. I've been developing RESTful web service, responsible for assembly and system composition of Intel's RSD. 
        Some of my work can still be found at github \emph{\href{https://github.com/intel/intelRSD}{intel/intelRSD}}.
        Below are some of my key achievements:
        \begin{missions}
        \item I've developed most of testing code \& libraries (\textasciitilde50\% of commits were mine)
        \item Implemented PODManager composition logic
        \item Implemented services discovery over network through DHCP
        \end{missions}
      }      
%----------------------------------------------------------------------------
\story{software engineer}{\href{https://www.finastra.com/}{Misys}}
      {2012 - 2014\\Gdynia, Poland}
      {c++, c, java,\\groovy, bash}
      {I've joined Misys to support Kondor+ product development.
        Apart from that, I've done some work around other products related to Kondor+
        like MarkitWire adapter, which gave customers ability to use Kondor+ together
        with Markit stack. I have also fixed several performance issues within Kondor+ product.
      }
%----------------------------------------------------------------------------
\story{software engineer}
      {\href{https://sii.pl/en/}{SII} (under contract to Intel Technology Poland)}
      {2011 - 2012\\Gdańsk, Poland}
      {c\#, tcl,\\python, bash}
      {I was hired to external team working for Intel Technology Poland. The team
        took over responsibility for \href{https://en.wikipedia.org/wiki/Mdadm}{\emph{mdadm}}
        validation, where Intel made support for their metadata. Later on I was
        responsible for Intel's iSCSI driver validation. Please find below some
        of my key achievements:
        \begin{missions}
        \item Implemented python-based testing framework for CIM RAID interface
        \item Developed dozen of mdadm tests
        \item Developed C\# library for storage discovery
        \end{missions}
      }
%----------------------------------------------------------------------------
\story{software engineer}{\href{https://micromade.pl}{Micromade}}
      {2010 - 2011\\Gdańsk, Poland}
      {c++, c, vcl}
      {At Micromade I was responsible for implementing brand new user interface of `bibinet` application.
        It is a management software for their well known product for work time tracking and access control.
      }
%----------------------------------------------------------------------------
\story{internship}{\href{https://voicelab.ai}{Voicelab}}
      {2009 - 2010\\Gdańsk, Poland}
      {matlab, c, php}
      {During my studies I was hired by Voicelab to develop several matlab simulations
        and php-based website utilizing noise reduction algorithms for sound recordings.
      }
%----------------------------------------------------------------------------
\section{Education}
\story{M.Sc., engineer}{Gdańsk University of Technology}{2006 - 2011\\Gdańsk, Poland}
      {Control Engineering\\and Robotics}
      {Faculty of Electronics, Telecommunications and Informatics:
        \begin{missions}
          \item Master thesis: \textit{Implementation of lattice adaptive filters on the CUDA platform}
          \item \href{https://projektgrupowy.eti.pg.gda.pl/editions/2/projects/496/posters/138}{\emph{Distinction of faculty dean for \itshape Anti-collision system for industrial robots}}
        \end{missions}
      }
\section{Skills}
\begin{skills} {Language}
\skill[Polish]{ Mother tongue }
\skill[English]{ Fluent }
\end{skills}

\begin{skills}{Programming}
  \skill[FP]{Utilize functional programming concepts to deliver reliable, composable software.}
  \skill[Languages]{Worked with multiple of technologies and programming languages.}
  \skill[Automation]{Never do the same manual work twice.}
\end{skills}

\begin{fields}{Scoring}{3}{5}
  \field{Scala}{5}
  \field{Functional Programming}{4}
  \field{Shapeless}{4}
  \field{Cats}{5}
  \field{build tools}{4}
  \field{fs2}{4}
\end{fields}

%\textit{I hereby give consent for my personal data included in my offer to be processed for the purposes of recruitment, in accordance with the Personal Data Protection Act dated 23.08.1997 (uniform text: Journal of Laws of the Republic of Poland 2002 No 101, item 926 with further amendments).}

\end{document}
